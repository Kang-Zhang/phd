\chapter{Validation of WARP3D \J Results with Abaqus} \label{chap:abaqus-validation}

% Here we document the solution of the FEACrack models inside Abaqus. Pretty good agreement except near \(\phi=0\).

In order to validate the WARP3D results developed in \Cref{chap:development-bending-models}, a subset of models were created for analysis in Abaqus.
The subset of models represent combinations of the most extreme geometry and material values:
\(\frac{a}{c} = 0.2\text{ and } 1.0\), 
\(\frac{a}{t} = 0.2\text{ and } 0.8\), 
\(E = 100\text{ and } 1000\), and
\(n = 3\text{ and } 20\).
Four \verb|.elt| files used in the WARP3D models for the combinations of \(\frac{a}{c}\) and \(\frac{a}{t}\) values were manually opened in FEACrack, and FEACrack's program options were modified to write Abaqus input files instead of WARP3D input files.
Material properties included in the input file would be modified in an Abaqus Python script.
FEACrack's application of traction conditions in WARP3D input files was not supported in Abaqus, so they were replaced with displacement values for the inner roller location extracted from WARP3D results.
Even if these values are not identical, the purpose of the validation is to compare the \J-CMOD curves calculated in WARP3D to those calculated in Abaqus.

Abaqus input files created from FEACrack are imported using the \verb|mdb.ModelFromInputFile| Python function, removing the need for the geometry creation and meshing effort detailed in \Cref{chap:app-quillen-rework}.
A sample mesh created by FEACrack is shown in \Cref{fig:abq_plate_ac02_at02}.
Elastic material properties were replaced with correct values for \(E\) and \(\nu\), and since Abaqus does not support a true linear plus power law material behavior, an extended range of stress-strain curve values were calculated to ensure no element stress or strain values exceed that range.

An analytically-rigid body was constructed a planar face at \(z=0\), and serves the same purpose as the elastic boundary plane from \Cref{sec:warp-elastic-boundary}.
Frictionless contact behavior was added to the Abaqus model, and a reference point on the rigid body was fixed in all degrees of freedom.
All but two load steps created by FEACrack were deleted, leaving only the ones defining symmetry, contact, and outer roller support conditions.
A new load step was created where a displacement boundary condition derived from the WARP3D displacement results was applied to all nodes coincident with the inner roller.
The final composition of an example bending model in Abaqus can be seen in \Cref{fig:abq_bc_ac02_at02}.
\begin{figure}[tbp]
\centering
\includegraphics[width=0.8\columnwidth]{{abq_plate_ac02_at02}}
\caption{\label{fig:abq_plate_ac02_at02} Example Abaqus bending model from FEACrack (\(\frac{a}{c}=0.2\), \(\frac{a}{t}=0.2\))}
\end{figure}
\begin{figure}[tbp]
\centering
\includegraphics[width=\columnwidth]{{abq_bc_ac02_at02}}
\caption{\label{fig:abq_bc_ac02_at02} Example Abaqus assembly and boundary conditions (\(\frac{a}{c}=0.2\), \(\frac{a}{t}=0.2\))}
\end{figure}

Final Abaqus files were written, and smaller models were tested interactively on a laptop virtual machine before the full set of 16 models was solved on TTU's HPC cluster.
Abaqus output databases were collected, data for \(z\) direction displacement of node 1 was used as a measure of one-half the experimental CMOD, and data for \J values at \(\phi = \) \SI{0}{\degree}, \SI{30}{\degree}, and \SI{90}{\degree} were extracted.
As seen in \Crefrange{fig:abq_wrp_ac02_at02_E0100}{fig:abq_wrp_ac10_at08_E1000}, Abaqus \J and CMOD values at \(\phi = \) \SI{30}{\degree} and \SI{90}{\degree} are nearly coincident with the values calculated in WARP3D, while the values for \(\phi = \) \SI{0}{\degree} are in close agreement except for cases of deeply highly elliptical cracks \((\frac{a}{c}, \frac{a}{t})=(0.2, 0.8)\) for any material and semicircular cracks for stiff materials \((\frac{a}{c}, E) = (1.0, 1000)\)\todo{Come up with better conclusions here.}.

\begin{figure}[tbp]
\centering
\begin{minipage}{0.45\columnwidth}
\includegraphics[width=\columnwidth]{{abq_wrp_ac02_at02_E0100_n03}}
\subcaption{\(n=3\)}
\end{minipage}
\begin{minipage}{0.45\columnwidth}
\includegraphics[width=\columnwidth]{{abq_wrp_ac02_at02_E0100_n20}}
\subcaption{\(n=20\)}
\end{minipage}
\caption{\label{fig:abq_wrp_ac02_at02_E0100} \J comparison between Abaqus and WARP3D (\(\frac{a}{c}=0.2\), \(\frac{a}{t}=0.2\), \(E=100\))}
\end{figure}

\begin{figure}[tbp]
\centering
\begin{minipage}{0.45\columnwidth}
\includegraphics[width=\columnwidth]{{abq_wrp_ac02_at02_E1000_n03}}
\subcaption{\(n=3\)}
\end{minipage}
\begin{minipage}{0.45\columnwidth}
\includegraphics[width=\columnwidth]{{abq_wrp_ac02_at02_E1000_n20}}
\subcaption{\(n=20\)}
\end{minipage}
\caption{\label{fig:abq_wrp_ac02_at02_E1000} \J comparison between Abaqus and WARP3D (\(\frac{a}{c}=0.2\), \(\frac{a}{t}=0.2\), \(E=1000\))}
\end{figure}

\begin{figure}[tbp]
\centering
\begin{minipage}{0.45\columnwidth}
\includegraphics[width=\columnwidth]{{abq_wrp_ac02_at08_E0100_n03}}
\subcaption{\(n=3\)}
\end{minipage}
\begin{minipage}{0.45\columnwidth}
\includegraphics[width=\columnwidth]{{abq_wrp_ac02_at08_E0100_n20}}
\subcaption{\(n=20\)}
\end{minipage}
\caption{\label{fig:abq_wrp_ac02_at08_E0100} \J comparison between Abaqus and WARP3D (\(\frac{a}{c}=0.2\), \(\frac{a}{t}=0.8\), \(E=100\))}
\end{figure}

\begin{figure}[tbp]
\centering
\begin{minipage}{0.45\columnwidth}
\includegraphics[width=\columnwidth]{{abq_wrp_ac02_at08_E1000_n03}}
\subcaption{\(n=3\)}
\end{minipage}
\begin{minipage}{0.45\columnwidth}
\includegraphics[width=\columnwidth]{{abq_wrp_ac02_at08_E1000_n20}}
\subcaption{\(n=20\)}
\end{minipage}
\caption{\label{fig:abq_wrp_ac02_at08_E1000} \J comparison between Abaqus and WARP3D (\(\frac{a}{c}=0.2\), \(\frac{a}{t}=0.8\), \(E=1000\))}
\end{figure}

\begin{figure}[tbp]
\centering
\begin{minipage}{0.45\columnwidth}
\includegraphics[width=\columnwidth]{{abq_wrp_ac10_at02_E0100_n03}}
\subcaption{\(n=3\)}
\end{minipage}
\begin{minipage}{0.45\columnwidth}
\includegraphics[width=\columnwidth]{{abq_wrp_ac10_at02_E0100_n20}}
\subcaption{\(n=20\)}
\end{minipage}
\caption{\label{fig:abq_wrp_ac10_at02_E0100} \J comparison between Abaqus and WARP3D (\(\frac{a}{c}=1.0\), \(\frac{a}{t}=0.2\), \(E=100\))}
\end{figure}

\begin{figure}[tbp]
\centering
\begin{minipage}{0.45\columnwidth}
\includegraphics[width=\columnwidth]{{abq_wrp_ac10_at02_E1000_n03}}
\subcaption{\(n=3\)}
\end{minipage}
\begin{minipage}{0.45\columnwidth}
\includegraphics[width=\columnwidth]{{abq_wrp_ac10_at02_E1000_n20}}
\subcaption{\(n=20\)}
\end{minipage}
\caption{\label{fig:abq_wrp_ac10_at02_E1000} \J comparison between Abaqus and WARP3D (\(\frac{a}{c}=1.0\), \(\frac{a}{t}=0.2\), \(E=1000\))}
\end{figure}

\begin{figure}[tbp]
\centering
\begin{minipage}{0.45\columnwidth}
\includegraphics[width=\columnwidth]{{abq_wrp_ac10_at08_E0100_n03}}
\subcaption{\(n=3\)}
\end{minipage}
\begin{minipage}{0.45\columnwidth}
\includegraphics[width=\columnwidth]{{abq_wrp_ac10_at08_E0100_n20}}
\subcaption{\(n=20\)}
\end{minipage}
\caption{\label{fig:abq_wrp_ac10_at08_E0100} \J comparison between Abaqus and WARP3D (\(\frac{a}{c}=1.0\), \(\frac{a}{t}=0.8\), \(E=100\))}
\end{figure}

\begin{figure}[tbp]
\centering
\begin{minipage}{0.45\columnwidth}
\includegraphics[width=\columnwidth]{{abq_wrp_ac10_at08_E1000_n03}}
\subcaption{\(n=3\)}
\end{minipage}
\begin{minipage}{0.45\columnwidth}
\includegraphics[width=\columnwidth]{{abq_wrp_ac10_at08_E1000_n20}}
\subcaption{\(n=20\)}
\end{minipage}
\caption{\label{fig:abq_wrp_ac10_at08_E1000} \J comparison between Abaqus and WARP3D (\(\frac{a}{c}=1.0\), \(\frac{a}{t}=0.8\), \(E=1000\))}
\end{figure}
